\chapter{Inter-branch quarrel in comparative perspective}
\label{ch:sopInPerspective}

%abstract
%This chapter provides a birds-eye view on how I study the legislative process in systems of separation of power (sop).  Two decision-making principles coexist in sop: one requiring the powers to concur on new policy, another establishing exceptions to concurrent consent.  There is a tendency in the literature to overlook exceptions (unilateralism), biasing appraisals of sop systems.  The chapter also incorporates these principles into an abstract framework – the amplified game of sop – which represents the essential rules of policy-making.  The components of this game are isolated to get a sense of the substantive claims that can be produced with each.  The chapter also introduces a discussion on the motivation that guides politicians in choosing their actions.  Motivation will be the clue to begin constructing a unified model of the legislative process in systems of sop. 

There is a classic distinction in comparative politics between two forms of democratic government, the parliamentary and the presidential.  The primary attribute of parliamentarism is that the legislative and executive powers of government are fused; under presidentialism there is a separation of these powers \citep{lijphart.1984}. There has been much controversy over the pros and cons of fusion and separation of power.  

Debate has taken place mostly at the theoretical level, with claims remaining, for the most part, unaccompanied by convincing empirical evidence.  But even on the theoretical front the state of the discussion is inconclusive, due in great part to the difficulty of isolating the independent effect that broad governmental forms have on government performance from those of a myriad of smaller institutional details.  Asking whether fusion is better or worse than separation of power may be too broad a question.  

My research branches out of this intellectual tradition.  I shall not, however, engage in a discussion of pros and cons.  In this thesis I focus, instead, on how the legislative process operates in systems of separation of power in the Americas.  Policy is, for the most part, written as a set of pieces of legislation.  The legislative process is consequently the primary arena where governmental actors make decisions affecting the citizenry.  

In this chapter I present an outline of my argument.  The chapter is divided into 9 short sections that are organized in the following fashion.  

Separation of Power.  Section 1 situates decision-making in a presidential or “separation of power” system in the context of the establishment of checks and balances in government, highlighting the dilemmas this raises for constitutional designers.  A definition of separation of power will be found in this section.  Section 2 introduces two principles of decision-making that, I argue, characterize any system of separation of power.  The principles are concurrent consent and unilateralism.  Section 3 offers a quick critical assessment of the relevant literature; I argue that there is a tendency, especially acute in the comparative literature, to omit the institutions of unilateralism in their conception of the legislative process.  

The Legislative Process.  Next, I concentrate on sketching the conceptual tools for the study of the legislative process in a generic separation of power polity.  Section 4 introduces a general or macro framework – the amplified game of separation of power – knitting together the theoretical parts of my thesis (as well as theoretical extensions that I will develop in the near future).  I then break the big framework into three smaller, analytically tractable, subsidiary models, all of which branch off from the encompassing frame (they are sub-games).  Sections 5, 6, and 7 provide outlines of the subsidiary models: the setter sub-game; the conditional decree sub-game; and the vetoes in the shadow of decrees sub-game.  These sketches introduce the basic building blocks with which my succeeding arguments are constructed, and in each of these three sections I also discuss the interesting substantive claims that are made based on this sort of model, foreshadowing what I will do in subsequent chapters or future extensions of the work.  

Actors' motivation.  Section 8 begins by synthesizing the stylized structures in which politicians interact in my model.  I then introduce a discussion on motivation: what is it that moves politicians who play these stylized games of the legislative process?  This brief discussion will anticipate the contents of the final chapter of the thesis, an attempt to synthesize two literatures on the legislative process (one in Anglo-American politics, another in comparative politics) that do not communicate despite dealing with essentially the same issues.  

Section 9 concludes by summarizing my approach to the study of the legislative process into a research design, presenting the questions that guide the thesis, defining the dependent and main independent variables, establishing a null hypothesis, and offering a general description of the methodology.\footnote{As will be evident, I have written this introduction in such a way that it addresses the contents of a large research agenda.  Some parts of the argument are not developed in chapters to follow; I do indicate, nonetheless, which pieces of argument are developed in subsequent chapters and which are projects.}

\section{What is separation of power?}
\label{s:whatSOP}

In framing a government which is to be administered by men over men, 
the great difficulty lies in this: you must first enable the government to 
control the governed; and in the next place oblige it to control itself
—Madison (1788, Federalist 51). \nocite{madison.1788}

Political theorists for at least two and a half centuries have been seeking institutional devices to constrain the capacity of government to violate the rights of the citizenry.  Three eminent figures in this school of thought, \citet{locke.1690}, \citet{montesquieu.1748}, and \citet{madison.1788}, favored the separation of policy-making power as an effective way to curb the all-too-human inclination of rulers to exploit the ruled --- ``il faut que le pouvoir arr\^ete le pouvoir'' suggested the second (163), which the third translated almost literally as ``ambition must be made to counteract ambition'' (322).  Yet these thinkers were also the first to recognize what their detractors have subsequently been emphasizing \citep{bagehot.1867,wilson.1884,romero.1893}: if separation of power results in an increase in the representativeness of policy it also implies an inevitable loss in the government's decisiveness – perhaps, in the spirit of Madison, the central dilemma in democratic theory.  

