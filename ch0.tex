\epigraph{\textbf{Duel}, \emph{n}.  A formal ceremony preliminary to the reconciliation of two enemies.  Great skill is necessary to its satisfactory observance; if awkwardly performed the most unexpected and deplorable consequences sometimes ensue.  A long time ago a man lost his life in a duel.}{Ambrose Bierce, \emph{The Devil's Dictionary}, 1911}

\chapter{Preface}
\label{ch:preface}

This is a dissertation about the legislative process under separation of power, employing presidentialism as a case study.  A president and an assembly that are formally independent from one another govern twenty-four out of every twenty-five citizens of the American continent.  Yet studies of how government decisions are made under such arrangement are confined, almost entirely, to the case of the United States.  A small number of studies are now attempting to generalize this single-case wisdom by adopting an explicitly comparative perspective.  This is the scheme that I follow in contrasting the presidential systems of the Americas in search of regularities in the way policy is made.  

The work centers on two features rendering separation of power effective and workable.  First is the veto power conferred to presidents and assemblies in the process of writing legislation.  These veto gates equip representatives with teeth to protect the interests of their constituents.  Second is the unilateral power to overcome a veto.  Unilateralism offers a way to unclog inter-branch imbroglios and stalemate, usually becoming effective only when special conditions are met (typically a super-majority, though not always).  

This structure for decision-making is then studied along with one key feature of any democracy, namely periodical elections.  The electoral connection meets politicians and their parties head on with the need to inject drama into their campaigns in order to make them more visible and effective in getting votes.  Instances of inter-branch conflict such as vetoes and veto overrides offer a special opportunity to advertise one's ideals in contrast to those of the opponents, and to do so not merely with words but with observable actions.  I claim that, often, politicians engage in veto politics in pursuit of votes in the next election.  

Presidentialism is a special case of separation of power.  My study of decision-making and posturing in the presidential systems of the Americas can be easily extended to the study of bicameral negotiations, of judicial-legislative and judicial-executive relations, and of bargaining between the different levels of a federal government. 

\begin{table}
\begin{tabular}{lrrrr}
                      & Bills           & Bills           &  Signed-        & Average        \\
                      & executive       & executive       &  to-vetoed      & yearly         \\
                      & signed          & vetoed          &  ratio          & vetoes         \\ \hline
California (1983--93) &  17,094         &  2,953          &   6:1           & 268            \\
Brazil (1985--96)     &    n.d.         &    870          &  n.d.           &  72            \\
New York (1983--93)   &   8,899         &    643          &  14:1           &  58            \\
Texas (1983--93)      &   7,253         &    258          &  28:1           &  23            \\
Florida (1983--93)    &   5,742         &    176          &  33:1           &  16            \\
Argentina (1983--97)  &   1,703         &    212          &   8:1           &  14            \\
Brazil (1946--64)     &    n.d.         &    260          &  n.d.           &  13            \\
U.S. (1945--92)       &  17,198         &    434          &  40:1           &   9            \\
Uruguay (1985--95)    &     954         &     50          &  19:1           &   5            \\
Chile (1990--94)      &     440         &     16          &  28:1           &   3            \\
Argentina (1862--76)  &   9,308         &    188          &  50:1           &   2            \\
Venezuela (1959--89)  &     850         &     20          &  77:1           & 0.5            \\
Mexico (1997--2000)   &    n.d.         &      0          &  $\infty$:1     &   0            \\
\end{tabular}
\caption{Executive signatures, vetoes, and summary statistics in selected cases (sorted by average yearly vetoes)}\label{t:aggVetoStats}
\end{table}

Consider executive veto incidence in selected systems (in Table \ref{t:aggVetoStats}) to get a preliminary impression of the frequency and cross-sectional variance in executive-legislative conflict.  The variance in the number of bills vetoed is substantial, a minimum of zero vetoes per year in Mexico, 1997--2000 to a maximum of nearly 270 in California, 1983--93.  In between are systems with high veto incidence (eg.\ Brazil, 1985--96), systems with medium incidence (eg.\ Florida, 1983--93 or Argentina 1983--97) to systems with close to nil veto incidence (eg.\ Venezuela).  I offer an appraisal of some factors to explain part of the variance.                                      
                  
\chapter{Acknowledgements}

I financed this dissertation with a portfolio of diverse research grants.  The University of California, San Diego most kindly provided the body of my funding.  The Center for U.S.-Mexican Studies offered me a Research Assistantship from 1995 until 1998, disbursing my tuition and fees and allotting me a monthly stipend.  The Department of Political Science gave me a second Research Assistantship under similar conditions from 1998 until the end of my residence at UCSD in year 2000.  A Friends of the International Center scholarship was granted to me in 1998.  The Center for Iberian and Latin American Studies Field Research Grant, in combination with a Dean's Social Science Travel Research Fund Grant, helped fund field research in the Fall of 1999.  I am grateful to all these branches of UCSD for their generosity.  

I also acknowledge Mexico's Consejo Nacional de Ciencia y Tecnología partial scholarship/credit for living expenses (media beca-crédito conacyt de manutención para estudios de posgrado sin beca-crédito para la colegiatura) from 1995 until 1999.  The Ford and McArthur foundations jointly complemented my stipend with another living-expenses grant for graduate studies from 1995 to 1997.  

I am also indebted to the faculty and peers at the Department of Political Science at UCSD for the invaluable support I received.  First and foremost I express my deepest gratitude to Gary Cox.  His continual guidance led my transformation from a student of politics to a professional researcher in Political Science.  Gary supported me since the very beginning at UCSD, patiently reading countless iterations of all my work with an incredible speed and incomparable clairvoyance.  Gary also gave me the privilege of being his co-author, and inspired my other major publication.  

I am also in debt with Mat McCubbins, whose seminars and workshops I attended year after year provided a fertile ground to develop the ideas that crystallized into this thesis.  Inseparable from this environment were Mat's students: Chris Den Hartog, Jamie Druckman, Andrea Campbell, Jenn Kuhn, and Greg Bovitz.  

The input of my dissertation committee members was no less pivotal in shaping amorphous ideas into a feasible research project.  My thanks and appreciation go to Paul Drake, Liz Gerber, Matthew Shugart, and Carlos Waisman.  

Wayne Cornelius turned the staff of the Center for U.S.-Mexican Studies into an invaluable resource at my reach throughout my residence at UCSD.  I acknowledge the confidence Wayne always had towards me.  My thanks also go to Kevin Middlebrook.  

Graduate school remains a highly communal endeavor despite the countless hours of solitary work.  Distinguished members of the community were Neal Beck, Gary Jacobson, Arend Lijphart, Victor Magagna, and Sam Popkin among the faculty; and Matt Baum, Emily Edmonds, Guadalupe González, Priscilla Lambert, Pablo Pinto, Marc Rosenblum, Dave Samuels, Jorge Schiavon, Dru Scribner, and Rafa Vergara among my student peers.  

Participants at the First International Graduate Student Retreat for Comparative Research, organized by the Society for Comparative Research and the Center for Comparative Social Analysis, UCLA gave me confidence that my project could fly outside UCSD.  Octavio Amorim Neto, Roberto Bavastro, Chris Den Hartog, Natalia Ferretti, Valeria Palanza, Pablo Pinto, Federico Estévez, and Jeff Weldon suggested ways to improve some part of my argument.  Alejandro Poiré offered me a space for 2000-01 at ITAM with a decent stipend and a reduced teaching load to complete the dissertation.  Gaby Cuevas, Vanessa Leyva, and Juan Navarrete provided research assistance for chapter 4.  

Fieldwork in Argentina and Chile would never have been as fruitful hadn't I received the disinterested help from a number of people.  Pablo Pinto's contacts became key coordinates to the political communities of Argentina.  Gisela Sin guided me through Buenos Aires and its academic and research milieu; she and Valeria Palanza kindly introduced me to the legislative process in Argentina.  Mariano Tommasi offered me a place to work at the Centro de Estudios para el Desarrollo Institutional; the research crew at CEDI rendered research in Buenos Aires much more pleasant and productive.  Mark Jones offered precise and decisive advice for the logistics of fieldwork in Buenos Aires (as well as incredible feedback for preliminary versions of chapters 4 and 5).  Gustavo Aparicio kindly provided me with a roof and a family throughout my stay in Buenos Aires.  Roberto Bavastro and Mario Maurich generously shared their data on rejected executive bills and on executive decrees, respectively.  I am also indebted to the personnel of the Argentine Congress' Departamento de Investigación Parlamentaria, who compiled for me much of the primary source evidence in chapter 4; Renée Silva, Alberto DiPeco, and María Isabel Giménez merit special mention.  Carlos Alberto Giacobone furnished me with the list of bloc leaders in the chamber of deputies.  My recognition also goes to Florencia Azubel and Carlos Toffoli, Julio Caviglioni, Constanza and Jorge Di Masi, Alberto Fohrig, Emilia and Edgard Mihailovitch, the late Guillermo Molinelli, Ana María Mustapic, Pablo \emph{Pebe} Pinto and María Susana Murgier, Ariel Puebla, Carlos Riera Cervantes, Catalina Smulovitz, and Federico Storani. 

In Chile I am grateful to Juan Enrique Vega, who generously allowed me to base my operations in Santiago and Valparaíso at Corporación Tiempo 2mil.  Verónica Kulczewki simplified the process of data collection and did a couple of key telephone calls on my behalf.  Peter Siavelis kindly offered me basic pre-field trip information.  Reynaldo Núñez Estrada, recommended me with the personnel of the Oficina de Partes of the chamber of deputies, which provided me with an electronic roster of bills in Chile.  Maya Fernández Allende guided me through Santiago, offering warm company.  I am also indebted to Isabel Allende, Neville Blanc, María Inés Bussi, Karin and Dennis Burnett, Francisco Geisel, and Alfredo Rehren for their help at one or another point.  

Thanks to family members for all their support: Val, Manuel, Jürgen, mother and father.  Last but not least, I declare my friendship to all those who made my years in San Diego enjoyable: Chavón and Cris; Big Red Abraham; Paulit; Emilushka and Gustavillo Amarillo; Maga and Quique; Daniela and Dwight; Arturo Gálvez; Pablísimo, Dolores, Paloma, and Joaquín; Marc, Kathie, Hazel, and Isaac; Sa\v{s}a \emph{und} Robert; Iván and Simone; Pedro Villegas; and Arunabh Lath. 

\chapter{Abstract}

I study the legislative process under the presidential form of separation of power (sop).  Three themes converge in my study.  First I ask why vetoes occur in sop systems?  In the U.S. the conventional answer is that vetoes are tactical maneuvers of normal democratic politics.  Vetoes are not harbingers of imminent democratic breakdown, nor evidence of gridlock.  Vetoes are part of everyday policy bargaining.  

Second, I ask whether the U.S. is an exception among presidential systems.  Does this logic apply to other cases such as those in Latin America?  Is it applicable to sub-national levels of government?  If vetoes are bargaining tactics then their use should vary with the institutional context.  

Third, I inquire about different explanations of veto incidence.  Are vetoes the product of incomplete information in bargaining hence mistakes by somewhat shortsighted politicians?  Are vetoes \emph{bargaining ploys} meant to build a reputation of toughness in light of asymmetric information?  Or are vetoes better understood as \emph{publicity stunts}, maneuvers aimed towards the public in search of support?  Incomplete information and position-taking compete to explain veto incidence.  

This dissertation reviews strands of literature that share the topic (sop) but not the audience (Anglo- and Latin-Americanists), offering a bridge over the gap.  I extend a formal model of executive-legislative bargaining (the agenda-setter model) to include a posturing motive for politicians to provoke vetoes.  The extended model is put to a test covering state-level governments of the U.S.  I then expand my research to Latin America, studying veto incidence in Argentina and three case studies of inter-branch bargaining in Argentina, Chile, and Mexico.  

Studies of presidentialism in the last decade have shifted attention from the dysfunctionality of gridlock (eg.\ Linz, Sundquist) to a wide range of tactical maneuvers that sop offers politicians (eg.\ Kernell, Cameron).  I follow the recent literature in analyzing a richer, livelier breed of executive-legislative relations.  Bully pulpit veto politics involve the presidency \emph{as well as} Congress.  
